\documentclass[sigconf, nonacm]{acmart}
%% The following content must be adapted for the final version
% paper-specific
\newcommand\vldbdoi{XX.XX/XXX.XX}
\newcommand\vldbpages{XXX-XXX}
% issue-specific
\newcommand\vldbvolume{14}
\newcommand\vldbissue{1}
\newcommand\vldbyear{2020}
% should be fine as it is
\newcommand\vldbauthors{\authors}
\newcommand\vldbtitle{\shorttitle} 
% leave empty if no availability url should be set
\newcommand\vldbavailabilityurl{}
% whether page numbers should be shown or not, use 'plain' for review versions, 'empty' for camera ready
\newcommand\vldbpagestyle{plain} 


%\newcommand{\red}{\textcolor{red}}
\newcommand{\techreport}{our technical report~\cite{techreport}}
\newcommand{\titlename}{Systems for Parallel and Distributed Large-Model Deep Learning Training}
% eat only in tech reports
\newcommand{\techreporteat}{}


\usepackage{hyperref,array,color,balance,multirow}
\usepackage{balance,float,url,amsfonts,alltt}
\usepackage{mathtools,rotating,amsmath}

\usepackage{etoolbox,listings}
\lstset{basicstyle=\ttfamily\tiny,breaklines=true}
\usepackage{bigstrut,morefloats,pbox}
\usepackage{graphicx,subfigure,xspace,verbatim,comment}
\usepackage{grffile}
\usepackage{ifpdf,fancyvrb}
\usepackage{algorithm}
\usepackage[noend]{algorithmic}
\usepackage{booktabs}
\usepackage{lipsum}  
\usepackage[all]{nowidow}
\usepackage{multirow}
\usepackage{pifont}
\usepackage{xcolor}

\newtheorem{theorem}{Theorem}[section]
\newtheorem{proposition}{Proposition}[section]
\newtheorem{corollary}[theorem]{Corollary}
\newtheorem{lemma}[theorem]{Lemma}
\newtheorem{definition}{Definition}[section]
\newcommand{\eat}[1]{}
 \newcommand{\eqdef}{=\mathrel{\mathop:}}
 \definecolor{mygreen}{HTML}{3EC300}
\newcommand\red[1]{\textbf{\color{red}#1}}
\newcommand{\xmark}{\color{red} \ding{55}}%

\DeclareMathOperator*{\gammasum}{\scalerel*{\Gamma}{\sum}}
\usepackage{scalerel}

%\DeclarePairedDelimiter{\ceil}{\lceil}{\rceil}

\newenvironment{packeditems}{
\begin{itemize}
  \setlength{\itemsep}{1pt}
  \setlength{\parskip}{0pt}
  \setlength{\parsep}{0pt}
}{\end{itemize}}

\newenvironment{packedenums}{
\begin{enumerate}
  \setlength{\itemsep}{1pt}
  \setlength{\parskip}{0pt}
  \setlength{\parsep}{0pt}
}{\end{enumerate}}

%\newcolumntype{P}[1]{>{\centering\arraybackslash}p{#1}}

\newtoggle{TR}
%\toggletrue{TR}

\DeclareMathOperator*{\argmin}{arg\,min}


\begin{document}\sloppy
\title{\titlename}



%\numberofauthors{8}
\author{Kabir Nagrecha}
\affiliation{%
\institution{University of California, San Diego}
}
\email{knagrech@ucsd.edu}



\begin{abstract}
Deep learning has transformed applications in a variety of domains, including computer vision, natural language processing, and tabular data analysis. The search for improved deep learning model accuracy has led practitioners to explore
increasingly larger model architectures, with some recent Transformer architecture designs using hundreds of billions of learnable parameters. The scale of these model architectures has introduced new systems challenges such as memory 
bottlenecks, poor runtime efficiency, and excessive computational costs of model development. Efforts to address these issues have explored techniques such as parallelization of neural architectures, spilling data across the memory hierarchy, and model compression. This survey will explore some key systems that incorporate these techniques, and evaluate their effectiveness in supporting large-model training. 
\end{abstract}
\maketitle

\pagestyle{\vldbpagestyle}
\techreporteat{
%%% do not modify the following VLDB block %%
%%% VLDB block start %%%
%%% VLDB block end %%%

%%% do not modify the following VLDB block %%
%%% VLDB block start %%%
\vspace{.3cm}
}
%%% VLDB block end %%%




\section{Introduction}
\label{sec:intro}
Over the last several years, deep learning (DL) has cemented itself as a critical component of a variety of high-value applications.
Computer vision (CV), natural language processing (NLP), and recommender systems now rely heavily on DL architectures to 
enable predictive modeling and analysis. The popularity of DL has led to the emergence of a new area, known as ``Systems for DL''
that studies the infrastructural and computational challenges introduced by the use of DL. 

\begin{figure}[th!]
\centering
	\includegraphics[keepaspectratio=true, width=\linewidth]{images/model_parallel_pipeline_parallel}
	\caption{Illustration of how state-of-the-art models in NLP, CV, and recommender systems have grown over time}
	\label{fig:scaling}
\end{figure}

Recent developments in DL practice have introduced a pressing new challenge in systems for DL research. Practitioners
have begun to explore the use of very large neural architecture graphs for DL models, with some containing billions and 
even \textit{trillions} of parameters! Key examples include the NLP models BERT-Large~\cite{bert2018}, GPT-3~\cite{gpt2020}
and Meta's deep learning recommender model (DLRM)~\cite{dlrm2019}. The sheer size of these models has introduced critical challenges in three key areas.

\begin{enumerate}
\item \textbf{Memory Scalability.} Standard DL training typically holds a model's parameters on the memory of an accelerator (e.g. GPU) and uses sampled data to compute gradient updates for each parameter. For a very large model, the space required to hold the parameters, intermediate computations, and gradient updates will typically exceed the relatively limited memory of an accelerator. High-end consumer-grade GPUs such as the Tesla V100~\cite{teslav100} have 16-32GB of on-device memory, but a large-scale DLRM might require hundreds of gigabytes of space.
\item \textbf{Performance.} Increased parameter counts are typically (but not always) correlated with higher execution times. Training a more complex model architecture can also require more data --- GPT-3 was trained on 300B tokens, for example~\cite{gpt2020} and the open-source BLOOM language model was trained on 366B~\cite{bloom2022}. Training such models can require weeks, or even months~\cite{dlrm2019,bloom2022}. As such, optimizations that can improve execution performance are highly beneficial
for developers of large-scale DL models.
\item \textbf{Cost of Training.} Standard solutions to the previous two challenges typically involve parallelizing storage or execution across multiple accelerators. However, this approach can drive up compute costs significantly. BLOOM was trained on 416 A100 GPUs, and Megatron-LM used 512~\cite{megatronlmblog2020}. This is not practical for most practitioners, particularly when these GPUs need to be reserved for a training period of weeks or even months. Reproducing BLOOM's training procedure on AWS
would cost 5.5 million dollars. Note that this does not even account for the added costs of model selection, wherein multiple models are trained to evaluate the best hyperparameter settings and configurations.
\end{enumerate}

Each of these challenges must be addressed for effective model development practices. Various systems and techniques have been developed tackling one or another of these problems. Some directions include data spilling/CPU offloading
~\cite{zero2019, zero2021,hydra2021,mpms2021,swapadvisor2021,l2l2020}, pipeline/model parallelism~\cite{gpipe2018,pipedream2018,terapipe2021,torchgpipe2020,megatronlmgpuscaling2021}, rematerialization~\cite{checkpointing2016}, and hybrid parallelism~\cite{flexflow2018,alpa2022,hydra2021,mpms2021,gshard2020}. Some recent works also factor in cloud computing costs~\cite{srifty2022,hydrozoa2022}.

Note that this survey will not cover \textit{graph neural network} training. Prior surveys~\cite{}

A few high-level surveys on large-model training have recently been released~\cite{surveylarge2022,lowmemory2019} but they are not comprehensive, and do not address some overarching challenges such as cost of training, model selection for large-models, and the intersection of techniques. This paper will address these critical areas and also go into further depth on state-of-the-art techniques.

This paper is organized as follows: Section~\ref{sec:background} goes over some necessary background and terminology. Section~\ref{sec:large_model} describes large-model architectures and some key examples. Section~\ref{sec:mlsys} explores the landscape of large-model DL training systems. Section~\ref{sec:future} explores possible future directions and emerging challenges. Finally, Section~\ref{sec:conclusion} concludes the survey and positions large-model DL systems training within the broader landscape of systems optimizations.






\section{Background and Preliminaries}
\label{sec:background}
We start with some background on key technical concepts and notation needed for the rest of this paper.
\subsection{Deep Learning}
Deep learning (DL) refers to a technique for approximating patterns within data by applying gradient descent to optimize the parameters of a directed graph of operators. In this paper, we refer to the DL model graph as a \textit{model}, or a \textit{neural computational graph}. Feedforward networks are a particular type of computational graph wherein inputs are continuously fed forward through a chain of operators, each of which is known as a layer. Most large-scale model architectures fall under this class.

Common DL operators include \textit{matrix multiplies}, \textit{differentiable activation functions}, and \textit{embedding table lookups} (functionally equivalent to matrix multiplies with one-hot vectors). The first two are used for directly transforming data inputs while the latter is used for associating some personalized vector with a categorical input (e.g. translating an application user into a vector). A DL model might use all of these operators together in a complex structure, typically known as a deep neural network (DNN).

The \textit{training} of a model involves an optimization procedure called stochastic gradient descent (SGD). It samples \textit{mini-batches} of data from the training dataset, then runs a \textit{forward} pass through the model graph. The forward pass transforms the input data features into an output target, then calculates a \textit{loss} value, or error, relative to some ground truth \textit{label}. This loss value is used to run a \textit{backward} pass, also known as backpropagation. Backpropagation refers to the process of repeatedly applying the chain rule to calculate updates for each model parameter that will minimize the output loss. Note that in order for this chain process to work, intermediate outputs, also known as \textit{activations}, between operators \textit{must be saved}. \textit{Inference} applications, wherein the model parameters is frozen deployed for prediction, can discard intermediate outputs when transforming inputs, but training has to save the intermediates (which can bloat memory costs).

A full pass of minibatch samples through the entire dataset is called an \textit{epoch}. Training a model typically requires many epochs of SGD. Popular DL tools implement many variants of SGD (e.g., Adam, AdaGrad, RMSProp, etc.), known as \textit{optimizers} but their data access patterns are identical. Some optimizers maintain their own parameters and state which are updated during execution.

The size of a DL model can be measured through the number of parameters and/or its memory footprint. The memory footprint of a model is generally correlated with the number of parameters. In order to compute a full minibatch-pass, intermediate data must be materialized between layers, and gradients will also need to be held in memory as they are chained backwards during backpropagation. As such, an upper bound of model memory footprint can be made based on the sum of its parameters' memory costs, its intermediates' memory costs, and its gradients' memory costs. This upper bound is generally a high overestimate, since most training frameworks will automatically discard intermediates and gradients as soon as they are no longer needed.

\subsection{Deep Learning Accelerators}
Matrix multiplies are typically the most common and computationally intensive operators within DL model graphs. Other operators exist (e.g. activation functions, embedding table lookups) but these are generally less computationally intense. The general matrix multiply operation, or GEMM, is a well studied target for parallelization~\cite{blislab2016}. As such, using hardware that can exploit this opportunity effectively is a common practice among DL model developers.

\textbf{GPUs} offer stronger parallelization capabilities versus CPUs for simple operations such as the GEMM. In particular, Nvidia GPUs support the cuDNN library~\cite{cudnn}, which implements low-level DL operators tuned for parallelization on GPU threads. However, GPUs must operate on data stored within on-accelerator memory, which is far more limited than standard system memory (DRAM). A state-of-the-art consumer GPU typically has less than 40GB of on-device memory, while DRAM capacity on a training machine might easily exceed 512GB. 

\textbf{TPUs}, or tensor processing units, are a new type of accelerator developed by Google specifically for DL. TPU cores are organized to support parallelism in GEMM operations and experimental evaluations show that it can achieve up to 10X faster performance than GPUs on some model architectures~\cite{tpubenchmark2019}. Unfortunately, TPU access is still very limited --- practitioners can only use one through Google's Cloud Platform. As such, this survey will primarily focus on systems targeting GPU execution.

\subsection{Parallelization Techniques}\label{sec:parallelization}
A variety of techniques exist for parallelizing DL training. Each offers different advantages or drawbacks, which tend to vary depending on the workload. Understanding the data access and communication patterns of each strategy is critical to evaluating them in the context of large-model training.

\textbf{Model parallelism} refers to the technique of partitioning, or \textit{sharding}, a neural architecture graph into subgraphs known, and assigning each subgraph, or \textit{model shard} to a different device. In a feedforward network, these shards might refer to groups of stacked layers.

\begin{figure}[th!]
\centering
	\includegraphics[keepaspectratio=true, width=0.9\linewidth]{images/model_parallelism_basic}
	\caption{An illustration of how a large feedforward network that does not fit into a single GPU could be model-parallelized over three GPUs to enable execution. Note that execution has not been sped up --- there is no parallel execution, only partitioned memory demands.}
	\label{fig:model_parallel_feedforward}
\end{figure}


The speedup potential of model parallelism depends largely on the architecture and the sharding strategy. Sequential model parallelism on a feedforward network, of the sort illustrated in Figure~\ref{fig:model_parallel_feedforward} will offer no scope for parallel execution, instead inducing a dependency graph between accelerators. This sharding strategy is still popular, however, as it can distribute memory demands across multiple devices and is fairly simple to setup. Other operators, such as embedding tables, are more amenable to being sharded width-wise as illustrated in Figure~\ref{fig:embedding_table_parallel}. Width-wise sharding strategies, more generally known as \textit{tensor parallelism}, can execute layers in parallel across multiple devices, but typically need to be followed by an all-gather communication pattern to produce a full output.

\begin{figure*}[th!]
\centering
	\includegraphics[keepaspectratio=true, width=0.9\linewidth]{images/embedding_table_parallel}
	\caption{An embedding table parallelized over 6 GPUs. Each GPU receives a different subset of the table's indices.}
	\label{fig:embedding_table_parallel}
\end{figure*}

Model parallelism of any sort introduces GPU-GPU communication. The latest Nvidia GPUs support ``NVLink'' interconnects --- high-speed GPU-GPU communication routes that offer as much as 900GB/s bandwidth --- which can help minimize overheads. NVLink is not always readily available, however, particularly when using cloud-provided machines that the user cannot customize easily. When NVLink is not supported, GPU-GPU communication runs over PCiE interconnects, which are much slower. The Tesla V100, generally considered a standard high-performance GPU for DL applications, supports a 16-lane PCiE 3.0 interconnect, at 16GB/s. 

To avoid having to transfer too much data over slow interconnects, model parallelism users generally aim to select a partitioning strategy that will minimize the size of activations that need to be transferred between shards, or else balance out computation to hide communication costs. Various sharding algorithms exist for this~\cite{flexflow2018,gpipe2019,lamp2020,mpanalysis2019,hydra2021,zero2019}.

\textbf{Data parallelism} is a common deep learning execution strategy that enables multiple mini-batches of data to be consumed in parallel. Data parallel execution techniques can be divided into two broad categories --- asynchronous data parallelism and synchronous data parallelism. 

The most well-known asynchronous technique is Parameter Server, wherein one core chief server holds a baseline set of parameters while distributed workers hold model replicas that train on different minibatches. The distributed workers occasionally send updates to the baseline server, which in turn will send out replacement parameters to the distributed workers to keep them updated. The workers can run out of sync with each other, as they only need to communicate/synchronize with the baseline server. Asynchronous techniques introduce many challenges, such as accuracy degradation versus single-worker training and irreproducible results due to variance in worker return times. For these reasons, asynchronous techniques are relatively uncommon in the modern DL training landscape. 

The most popular synchronous data parallel execution technique is Distributed Data Parallelism (DDP). DDP replicates a model and assigns copies to $m$ different accelerators. These replicas consume different minibatches of size $n$ in parallel and produce gradient updates for their local replica. These gradients are then aggregated across replicas to produce a global update, typically using an all-reduce communication pattern. This global update is then applied to every replica in parallel. This technique is mathematically equivalent to single GPU training with batch size $m \times n$. While this technique induces an all-reduce communication step, these overheads can generally be overlapped and hidden under model execution times.

\textbf{Hybrid parallelism} refers to strategies which \textit{combine} different parallelization strategies to achieve higher overall performance. For example, overlaying data parallelism on top of model parallelism could enable a user to achieve memory scalability across multiple devices along with the execution speedups of data parallelism. These strategies have tradeoffs that need to be factored into their design. In a simple overlay hybridization, model parallelism's multi-device requirements are multiplied by the replication requirements of data parallelism. A further overlay of task parallelism (e.g. in multi-model training) could add another multiplication factor into the equation. More complex hybridizations are covered in Section~\ref{sec:mlsys}.

\subsection{Model Modification}
Another approach is to actually reduce or alter the model itself to reduce its memory demands. Various techniques for \textit{compression} exist, some dynamic that readjust the architecture during training, and other static ones that reduce the model up front. Typically, compression induces an accuracy-size tradeoff, where higher compression factors reduce model learning capacity. This tradeoff can make this technique ill-suited to accuracy-critical applications, though well-managed and strategic compression can minimize the accuracy loss.

\textbf{Model distillation} attempts 


















\vspace{-2mm}
\section{Large Model Architectures}\label{sec:large_model}
To clarify the need for large model training systems and motivate their design, we describe various large model architectures and the unique challenges each presents. We can broadly classify the scale of model architectures under two categories --- depth-wise scaling and width-wise scaling. depth-wise scaling is most commonly needed for long, sequential chain architectures like Transformers. width-wise scaling is commonly used for very wide, easily parallelized operators (e.g. table lookups). A third setting, example scaling, describes the setting where the size of input samples drives the scaling challenge rather than the size of the model.This setting does not fall under the scope of ``large-model training'', so we do not discuss it in-depth in this paper.

\subsection{Deep Models \& Transformers}
Transformers, the most common example of very deep model architectures, have recently become very popular in a variety of domains. Benchmark tasks in natural language processing (NLP), computer vision (CV), and even tabular data analysis are now led by Transformer architectures. These models consist of stacked ``self-attention'' blocks, each of which consists of multiple dot product operations and matrix operations. Practitioners have found that increasing the depth of Transformers by stacking on more attention blocks generally improves accuracy~\cite{transformerStack}. As such, Transformers have motivated practitioners to explore increasingly deep model architectures. 

These models present a critical challenge for training. One of the first large Transformers was BERT-Large~\cite{bert2018}, using 345M parameters. The memory demands of training this model with a reasonable minibatch sample size on a GPU already required practitioners to use high-end GPUs. Since then, Transformers have only become deeper and deeper --- some recent ones have reached one trillion parameters, demanding space well beyond the memory capacity of \textit{any} GPU on the market. 

As such, techniques such as model parallelism are essentially necessary for large Transformer training, and deep model training in general. However, enabling parallel execution for very deep models can be challenging. The most natural sharding strategy for a deep sequence of layers is to partition the sequence into subsequences. But this approach forces the user to add GPUs without actually benefiting from any performance benefits ---- partitioning a sequence into subsequences does not offer any opportunities for parallel execution speedups. Consider a trillion parameter model that needs to use 1024 GPUs to even fit in memory. All of these GPUs are only being used to ``enable'' execution, and provide no performance benefits. In fact, the strategy would likely be slower than an equivalent in-memory training job due to the inter-GPU communication costs.

Some strategies for width-wise sharding exist, such as parallelizing operations in attention blocks across multiple GPUs. However, these approaches require more customization, add communication overheads, and require substantial effort on the part of the model designer to implement. As such, most systems for deep model training prefer to apply a generalized depth-wise sharding strategy that can be optimized for all deep model classes rather than targeting a single architecture at a time.

Despite the challenge of sequential dependencies, depthwise-sharding can introduce many opportunities as well. Techniques such as \textit{pipeline parallelism} and \textit{spilling} only work on depth-wise sharded models. We expand more on these techniques in Section~\ref{sec:mlsys}.

\subsection{Wide Models \& Embedding Tables}~\label{sec:embedding}
Wide models present a very different set of challenges from deep models. While it is easier to parallelize them in a performance effective manner (widthwise rather than depthwise), widthwise partitioning generally requires all-gather communication steps to aggregate parallelized partial outputs.

 \textit{Embedding tables} in recommender models are generally the most popular candidates for width-wise sharding. Embedding-based recommender models are used at most companies which gather entity-specific data (e.g. Meta, Netflix, TikTok) to create customized experiences. A standard approach is to create a table that maps user IDs to trainable vectors that can then be fed to some other DNN placed on top. However, for this to work on a multi-billion user platform such as Facebook, the corresponding table has to be very wide. A three billion index table with size 1024 trainable vectors filled with floats would require 12TB of memory. A real-world recommender might include \textit{multiple} such tables for different lookups (e.g. user table, business table, video catalog table), further increasing memory costs. 

Partitioning an embedding table is an easy task, given that table lookups are embarrassingly parallel --- a lookup on one index does not rely on other indices. As such, sharding a table into subtables assigned to different GPUs is a common strategy to distribute memory costs. Parallel execution across shards can easily be achieved by simply routing index lookup requests in a minibatch to the appropriate GPU. However, in order to re-aggregate the minibatch after the parallel table lookups (e.g. to feed to the top DNN), a potentially expensive all-gather communication step is necessary.

It is less common (but not unheard of) to apply width-wise sharding to other operators (e.g. very large matrix multiplies) can also be parallelized in a width-wise fashion. In general, embedding tables are the most memory-intensive single operators~\cite{dlrmscale2020}. Given that the primary use-case for width-wise sharding is embedding tables, optimizing for this case may seem overly specific. However, embedding tables and recommender models make up an outsized proportion of DL workloads --- Meta reports that 50\% of their DL training cycles are spent on embedding-table-based recommender models~\cite{dlrmscale2020}. As such, optimizing for the very wide model case is well worth it even if the applicability is more limited than optimizing for sequential deep model scalability.
\vspace{-2mm}
\section{Large Model Training Systems}\label{sec:mlsys}
Several systems have emerged to address the challenge of enabling efficient large-model training. We now describe the major classes of large-model training systems as well as key instances of each category. In general, most of these systems target a specific setting of large-model training (e.g. sequential deep models, wide models) as described in Section~\ref{sec:large_model}.

\subsection{Basic Techniques}
A few basic techniques such as rematerialization are often used as common building blocks for more advanced large-model training systems. In general, these techniques have minimal impact on organization and structure, making them amenable for integration with other approaches.

\subsubsection{Rematerialization}
Rematerialization, also known as gradient checkpointing, attempts to minimize the memory demands of \textit{backpropagation} specifically~\cite{checkpointing2000, checkpointing2016}. As explained in Section~\ref{sec:background}, backpropagation requires saving intermediate operator outputs for proper application of the chain rule for gradient computation. However, intermediate output tensors can demand a great deal of memory!  Some analyses~\cite{lowmemory2019} have shown that activations make up as much as 95\% of memory consumption for ResNet~\cite{resnet2015} and as much as 80\% of memory usage for some Transformers. Rematerialization trades compute for memory by initially discarding most of the activations except for a few \textit{checkpoints}, then \textit{recomputing} the discarded during backpropagation using the checkpoints. In this way, only the intermediates between checkpoints need to be stored in memory at any given point.

This approach does induce computational overhead --- the forward pass is effectively being run \textit{twice}. However, the operators in the forward pass are generally faster than the automatic differentiation procedure used in backpropagation, so the overhead is smaller than it might seem. Some gradient checkpointing systems claim to have only 30\% overheads for 6-7X memory savings~\cite{checkpointing2016}.

Checkpointing is critical to techniques such as \textit{pipeline parallelism} and \textit{shard alternator parallelism}.

\subsubsection{Accumulation}\label{sec:accum}
Accumulation targets the memory demands of batched \textit{gradients} in backpropagation~\cite{gpipe2019}. Section~\ref{sec:background} describes how stochastic gradient descent batches samples into minibatches that are fed through the model. In turn, we can consider the gradients that are produced for parameter updates to be the aggregation of the updates that would have been applied for each sample. Accumulation delays the application of these aggregated gradients, instead computing new minibatch-gradient-updates and accumulating them onto our aggregated gradient vectors. The new gradient is now the aggregated sum of \textit{2} minibatch updates, rather than 1. In this way, we can scale up our effective minibatch size and gradient impact without actually training a larger batch. We refer to the smaller, individual batches as \textit{microbatches}, and keep referring to the effective summed batch as the minibatch. 

Accumulation is essential to \textit{pipeline parallelism}, and is often used in conjunction with other techniques.

\subsubsection{Low Precision Representations}
Most training frameworks (e.g. TensorFlow, PyTorch)\red{Cite them} use single-precision float (32 bit) representations of gradients and parameters. Double-precision representations (64-bit) are relatively uncommon. One way to reduce the memory demands of training a model is to use \textit{half-precision} (16 bit) representations of data. Naturally, this induces an accuracy loss as values are being approximated. However, this approach can offer both speedups and memory savings. To try and balance this, \textit{automatic mixed precision}~\cite{amp2020} will automatically try and determine when data can be safely compressed to 16 bit without accuracy losses. AMP generally reports little-to-no accuracy losses while achieving as much as 5.5X speedups when training large models~\cite{amp2020}. Since AMP is directly modifying values at a very low-level, this technique is generally orthogonal to actual systems approaches for large-model training. 

\subsubsection{Sparse Representations}
In some cases, the vectors used in DL training are very sparse. As an example, embedding table lookups generally only involve a few indices of the table. The gradient vector applied to the table will only have non-zero values at the used indices, while the rest of the gradient will be zeroed out. Actually maintaining all of these zeroes in memory is unnecessary and wastes memory. Sparse representations attempt to compress these vectors down to their non-zero values while avoiding any information loss. The most simple approach, commonly used by default for embedding tables, is to represent a gradient as key-value pairs mapping indices to gradient values. An example is illustrated below.

\[ <0, 0, 0, 0, 0, 0, 4, 0> \rightarrow \{6: 4\} \]

This representation can easily be mapped back for application while discarding unnecessary zero values.

Some complications arise when combining sparse representations with operations that assume standard vector representations, such as all-reduce communication patterns. Some works~\cite{mlplatformmeetup2022} show how this can be resolved by more complex communication patterns or as-needed conversions back to standard representations. Sparse vector representations address a very specific problem, but are critical for efficient training of some operators such as wide embedding tables.

\subsection{Pipeline Parallelism} \red{add citations}
Pipeline parallelism targets the ``sequential deep model'' setting. It is a direct extension of the model parallel training paradigm described in Section~\ref{sec:background}. Model parallelism creates a staged-out sequence of shards, creating a natural ``pipe'' structure. Pipelining simply exploits this pipe structure by attempting to fill up the stages with operations, reducing the idling that sequential model parallelism suffers from. Consider the example of a pure feedforward network, like the one illustrated in Figure~\ref{fig:model_parallel_combined}A). Each shard can be considered a stage of the pipe, such that a model partitioned three-ways over three GPUs is now a three-stage pipeline.

In CPU pipelining, we fill up the pipeline with various instructions being sent to the CPU\cite{shen2013modern}. For DL pipelining, we fill up the pipeline with microbatches, like those used in gradient accumulation. In essence, pipeline parallelism is the combination of gradient accumulation and model parallelism. Independent microbatches are shuttled through the shard pipeline, then gradients for each microbatch are accumulated for each pipeline stage. Once the gradients for the full minibatch (combination of all microbatches) are all aggregated, they can be applied to the model.

Backpropagation presents a challenge for pipeline parallel training. As explained in Section~\ref{sec:background}, intermediate outputs must be available for backpropagation to occur. When combined with accumulation, however, this would require us to store a different intermediate output set for each microbatch, thus robbing us of any scalability advantage offered by accumulation. GPipe~\cite{gpipe2018}, one of the first pipeline parallel training systems, proposed combining accumulation with checkpointing to address this issue. Activations would only be stored at shard/pipe stage boundaries, with recomputation occurring as gradients shifted backwards through the pipe during backpropagation. The checkpointing approach is now standard in most, if not all, pipeline parallel training systems.

Another challenge is presented by the structure of the pipe. For pipelining to work, the shard pipe must be bidirectional. Input and activations flow forward during prediction, and gradients flow backward during backpropagation. This leads to a problem --- data within the pipe will ``collide'' on stages as it flows in both directions. As such, a pipeline flush occurs between prediction and backpropagation. The flush can severely hurt performance if not properly managed. Figure~\ref{fig:pipeline_parallel} illustrates a pipeline parallelized model. Note that all three accelerators are only active on steps 4 and 10 --- the rest of the time at least one accelerator is idle.

There have been many attempts to address this issue. GPipe~\cite{gpipe2018} suggested increasing microbatch counts while keeping accelerator counts constant, so the pipe could stay full for longer. This would not \textit{eliminate} the flush, but it would improve overall efficiency. However, this approach would demand more memory to store more checkpointed microbatch activations as well as larger input batches (potentially affecting convergence).

Another solution was proposed in the form of \textit{asynchronous pipelining}, which would reorder pipeline stages and backpropagation to eliminate the flush. Rather than running the stages for one microbatch in a sequence, they might run for one stage then be paused while a different microbatch runs a different stage. This ``decoupling''  of the order relaxes the problem into a more efficient one --- at the cost of affecting data consumption ordering and consumption. The 1F1B pattern runs one forward stage for every backward stage (on different microbatches) to maintain a perfect ratio and utilization. While asynchronous pipelining can perform well, it should be noted that it is not a general solution --- the accuracy losses are case-specific and can often be substantial. Applications where accuracy is critical and convergence behaviors must be replicable (e.g. model selection) are not a good fit for asynchronous pipelining.

\begin{figure*}[th!]
\centering
	\includegraphics[keepaspectratio=true, width=\linewidth]{images/model_parallel_pipeline_parallel}
	\caption{An illustration of pipeline parallelism over three shards. The input minibatch is partitioned into \textit{microbatches} that are then shuttled through the pipe stages. Before backpropagation, the forward prediction pipe has to fully clear, leading to idling. All microbatches must be backpropagated prior to gradient application for mathematical equivalence, as explained in Section~\ref{sec:accum}. \red{Shrink Image to make text legible}}
	\label{fig:pipeline_parallel}
\end{figure*}

\subsection{Memory Offloading \& Spilling}\label{sec:spilling}
While model parallelism looks at execution over multiple GPUs to distribute memory demands, some systems attempt to make use of main system memory (DRAM) rather than horizontally scaling across more GPUs. The primary motivation for this approach is that while GPU memory is limited and expensive, DRAM is substantially cheaper and readily accessible.

Consider an AWS-provided p3.2xlarge node. It has a single V100 GPU with 16GB of on-device memory. A large DL model might not fit into the node's GPU memory, but in actuality the node has far more \textit{total} memory still available --- 61GB of DRAM plus the 16GB of GPU memory. DRAM (sometimes referred to as CPU memory in the ML systems literature~\footnote{We consider the term CPU memory is ambiguous, given that it could be interpreted to refer to registers or CPU caches. In this paper, we use the term DRAM to refer to main system memory, but it should be noted that other ML systems papers may use the phrase CPU memory to refer to the same concept.} is far cheaper and more available than GPU memory, and a standard cloud-provided multi-GPU machine might easily have hundreds of GBs of DRAM. 

If a model's memory demands can be spread across both DRAM and GPU memory, a large model could be trained without the need for model parallelism's multi-GPU costs. A on-demand AWS p3.2xlarge node offers 77GB aggregate memory at a rate of \$3.06 per hour. In order to have the same amount of GPU-only memory, a p3.16xlarge 8-GPU node would be necessary --- costing the user \textit{8X as much} at \$24.48 per hour.  Note that this is not an exact comparison --- the p3.16xlarge node actually has 128GB of memory rather than 77 due to how AWS allocates DRAM-GPU count ratios, and GPU memory does not perform in precisely the same way as DRAM. But the general point is clear --- for pure storage purposes, making use of DRAM is far more cost-effective than scaling up to more expensive higher-GPU-count machines. As such, offloading part of the memory demands of a model to DRAM rather than scaling across multiple GPUs can be an attractive option for cost-conscious users (e.g. small enterprises and researchers). 

Many initial works~\cite{tflms2019,meng2017,swapadvisor2021,vdnn2016,wang2018} treated offloading as a ``swapping'' problem --- deciding when to swap tensors off of GPU memory and onto DRAM. Most use graph analysis algorithms to determine where to ``inject'' a swap operation based on when an activation, gradient, or parameter might next be used in the execution graph. SwapAdvisor, the most advanced of these swapping systems, uses a parallelized genetic search algorithm to analyze where the swap operators should be placed for best performance. It was also one of the first systems to support offloading \textit{parameters} as well as activations, which is critical for training recent large model architectures which use billions of parameters.

These complex swapping procedures can be difficult to setup --- SwapAdvisor's search algorithm takes roughly an hour to complete. Moreover, they are difficult to extend to multi-GPU training, as there is no clear way to extend the swap-injected graph technique to cover multi-GPU parallelism.

Another approach was proposed with ZeRO-R~\cite{zero2019}, a system for offloading that sends activations and parameters to DRAM dynamically. This approach ``offloads when needed'', rather than planning offloads up front. The irregularity of the design can introduce issues such as memory fragmentation, but it adds a great deal of flexibility versus graph-based designs. A later version, ZeRO-Infinity~\cite{zero2021} extended this to offloading to NVMe/disk storage for further scalability.

\begin{figure}[th!]
\centering
	\includegraphics[keepaspectratio=true, width=\linewidth]{images/model_spilling}
	\caption{Hydra's spilling strategy simply promotes and demotes model parallel shards on and off of GPU memory. Other spilling designs such as the one used by ZeRO-Offload are similar, though less strictly structured.}
	\label{fig:shard_parallel}
\end{figure}

Hydra~\cite{hydra2021} opts for an ``independent block'' strategy, dividing a model architecture into submodels (like model parallelism) that can then be spilled between DRAM and GPU memory freely. An analogy can be drawn to spilling in RDBMSs, where independent data chunks can be sent down to a lower level of memory. Unlike other spilling systems, Hydra's execution pattern is identical to model parallelism, and separates the execution of each model shard entirely. It still tries to overlap communication and computation, but ignores the complexities of fine-grained tensor offloading explored by other CPU-offloading techniques. This generalization makes it less-than-optimal for single-GPU execution, but makes it far more amenable to hybridization with multi-GPU parallelization techniques. We expand on this further in Section~\ref{sec:mt_parallel}.

L2L~\cite{l2l2020} uses a design similar to Hydra's but is more restricted in its sharding approach. It targets Transformer architectures specifically, and swaps across self-attention blocks with heuristics selected specifically for its target class of models. This allows it to perform very well on Transformer architectures, but prevents it from achieving the flexibility of Hydra or the dynamic generality of ZeRO-R.

Note that these techniques are generally used for distributing \textit{depth-wise} large model memory demands, as they all exploit some kind of sequential ordering in execution. A very wide operator (e.g. an embedding table) that cannot be serialized without substantial performance slowdowns, cannot easily be spilled across DRAM and GPU memory. The only option for hybrid-device execution on wide operators is to either serialize the parallel operator (index lookup in the table case) and rewrite the series of operations into a deep, rather than wide, model, or else to actually execute the wide operator on the CPU. 

Some systems combine offloading with mixed CPU-GPU computation. In general, it is preferable to run a model entirely using GPU or TPU compute, as most DL operators will run much faster on accelerators that support high degrees of parallelism. In the memory-limited case, however, it may become more practical to run some memory-intensive operations on the CPU using DRAM storage for operands rather than executing everything on the GPU.

ZeRO~\cite{zerooffload2021} proposed running parameter updates on the CPU while GPU execution is ongoing, specifically for the popular Adam optimizer~\cite{adam2014}. The Adam optimizer holds some state parameters (typically 32-bit) and needs to run on 32-bit parameters to avoid accuracy degradation. Unfortunately, this prevents users from exploiting 16-bit representations for reduced memory demands. The ZeRO version of the Adam optimizer maintains 32-bit versions of the parameters on DRAM and low-precision 16-bit versions on the GPU to consume less memory. During execution, the system spills gradients and optimizer state onto DRAM, then runs parameter updates on the 32-bit parameters using \textit{CPU processing}. The updates are propagated back to the 16-bit parameters in a secondary step that overlaps CPU-GPU communication with GPU computation. 

Mixed-CPU-GPU compute is also common for very large recommender models. As explained Section~\ref{sec:embedding}, embedding tables are very wide memory-intensive operators which generally feed into some smaller DNN for further processing. Without any optimization, the sheer scale of the embedding table would force CPU-only execution~\cite{dlrmscale2020}. Alternatively, a user could place the embedding table on the CPU while the DNN sits in GPU memory and enjoys the benefits of GPU acceleration. Some works such as Hotline~\cite{hotline2020} try and pipeline data through the model, from the CPU-based embedding table into the GPU-accelerated DNN. They demonstrate that this mixed compute approach can be even faster than width-wise multi-GPU model parallelism, as it eliminates the need for the all-to-all communication step described in Section~\ref{sec:embedding}.

\subsection{Hybrid Parallelism}
The basic parallelization techniques described in Section~\ref{sec:parallelization} can be combined in different ways. Various systems have attempted to combine the benefits of the various ``basic'' parallel execution approaches (e.g. data parallelism, model parallelism) to offer users higher performance and scalability. Hybrid parallelism techniques can be classified into two broad categories --- ``true'' hybrids that integrate parallelization techniques from the ground up, and top-down hybrids that select between different strategies at different stages of execution. 

\subsubsection{Ground-Up Hybrids}\label{sec:mt_parallel}
Traditionally, combining model parallelism with other techniques from the ground up has been a challenging task. Model parallelism drives up the GPU requirements of standard execution, which can make combinations with replication-based or multi-instance techniques for parallelism (e.g. data parallelism, task parallelism) impractical as they scale up model parallelism's device requirements further still. 

To address this problem, Hydra~\cite{hydra2021} proposed using a spilling technique like those outlined in Section~\ref{sec:spilling} to reduce the number of GPUs needed for scalable model-parallel training, and then applying a layer of task parallelism on top to support efficient multi-model training. The Hydra system then exploits the segmented nature of model parallelism to enable hybrid ``fine-grained parallel'' schedules that can outperform both standard task parallelism and model parallelism. Figure~\ref{fig:shard_parallel} illustrates. At the moment, Hydra is the only system to explicitly target the multi-model setting for very large models, but this area will likely grow in importance as practitioners struggle with the costs of model selection and multi-user cluster management.

\begin{figure}[th!]
\centering
	\includegraphics[keepaspectratio=true, width=\linewidth]{images/shard_parallel_schedule}
	\caption{Hydra's ``Shard Alternator Parallelism'' combines sequential model parallel spilling and task parallelism to outperform both standard model and task parallel techniques.}
	\label{fig:shard_parallel}
\end{figure}

Fully Sharded Data Parallelism (FSDP), originally introduced with ZeRO~\cite{zero2019} offers a hybrid of model parallelism and data parallelism. Unlike Hydra, which still executes in a model-parallel-sharded fashion, FSDP only uses model parallelism to \textit{distribute} the model over data parallel instances with each data parallel clone holding a partial set of parameters for each layer. When a layer is executed, FSDP runs an all-gather step to produce the full, unsharded layer on every data parallel instance. The layer is then executed in a purely data parallel fashion. After the layer is executed, it can be resharded immediately after to redistribute the memory footprint. A similar approach is used for backpropagation.



Under FSDP, the per-accelerator memory requirements are reduced down to the footprint of a single layer plus the partitioned memory requirements of the rest of the layers. Discarding the single layer requirement as a constant factor, we can represent this as an $O(n/k)$ reduction, where $n$ is the original model memory footprint and $k$ is the number of data parallel instances. This enables users to simultaneously benefit from the performance of data parallelism and the scalability of model parallelism. Note that this does add substantial communication overheads --- an all-gather is run for every layer --- and still requires horizontal scaling for scalability, unlike spilling-based techniques.

ZeRO-Offload~\cite{zerooflload2021} proposed combining FSDP with spilling per-accelerator, offloading sharded layer parameters that will not be used in the near future. This offers substantially better scalability, but introduces more communication overheads through CPU-GPU communication. ZeRO works to overlap the communication with compute, but some slowdown is generally inevitable. Analyses have shown that FSDP is slower than standard data parallelism (though more scalable and capable of running substantially larger models). Proponents of FSDP claim that users can exploit its higher scalability to increase batch sizes (and thus bring execution times in line with DDP performance), but we note that batch size can affect accuracy convergence behaviors. Scaling batch size for better FSDP performance can lead to the same issues that we outlined in our discussion of asynchronous pipelining (though less extreme).

3D Parallelism combines FSDP with pipeline parallelism and tensor parallelism to exploit scalable data parallelism along with parallel depth-wise and width-wise sharded execution. This usually takes the form of applying FSDP in some parts of the model, pipelining in another, and tensor parallelism in another segment more amenable to width-wise sharding. 3D parallelism generally requires a great deal of customization based on the model architecture --- it cannot be applied out-of-the-box like Hydra or FSDP. That being said, it has been applied successfully in the training of many very large-scale models such as Megatron-LM~\cite{megatronlmblog2020} and BLOOM~\cite{bloom2022}. In the future, a new ``4D parallelism'' might be possible by combining 3D parallel hybrids with Hydra's sharded task parallelism.

\subsubsection{Strategy Finding}\label{sec:hybrid_parallel_strategy_finding}
Strategy-finding systems attempt to automate the process of combining parallelization techniques within a model. A few recent examples are FlexFlow~\cite{flexflow2018} and Alpa~\cite{alpa2022}. 

FlexFlow, which was built prior to the development of advanced DL parallelization techniques such as pipeline parallelism, FSDP, and sharded task parallelism, only explored data, tensor, and model parallelism, and primarily targeted convolutional neural networks. FlexFlow builds a device topology graph that models accelerators as nodes and interconnects (e.g. NVLink, PCIe, Infiniband network) as edges. This allows it to produce hybrid parallel execution strategies that account for the cost of data movement between edges in a given device configuration. It uses a simulator to evaluate different partitioning strategies, using pilot passes to model operator runtimes and theoretical calculations based on edge bandwidths to model communication overheads. Using the simulator as an oracle, it evaluates different ways to partition the operators. Note that this ``partition'' based representation of parallelism cannot support parallelization techniques that exploit independent execution on different tasks (e.g. task parallelism, pipeline parallelism), though it could potentially support FSDP. In addition, it does not explicitly account for memory scalability or the possibility running out of device memory in a particular configuration~\cite{memflow2020}.

Alpa 	accounts for memory scalability more explicitly, and considers inter-operator parallelism (e.g. model parallelism, pipeline parallelism) rather than just intra-operator partitioning like FlexFlow. It uses an ILP formulation to determine how to setup the parallelization strategy, then modifies the execution plan if a stage will exceed device memory limits~\cite{alpa2022}. This approach can enable better performance than FlexFlow by accounting for a broader strategy search space.

These hybrid parallelization strategies are well suited to static, non-data-dependent execution tasks (e.g. non-recurrent neural networks). However, they do not scale well to more dynamic tasks such as multi-model training --- they are compilers for training, not schedulers. Future works could consider bridging this gap, building a dynamic hybrid parallel executor.

\subsubsection{Model-Data Parallelism for Recommender Models}
DLRMs present a unique challenge to practitioners because they combine two different scaling challenges. The embedding tables are very wise, and warrant width-wise partitioning for tensor parallel execution. The top DNN is computationally intensive but small, and would benefit most from data parallelism.  As such, a hybrid strategy that applies tensor parallelism to the table of the model and data parallelism to the DNN would perform well on recommender models. This approach has become the standard for fully GPU-accelerated DLRM training~\cite{dlrm2019} though the heterogeneous CPU-GPU execution that we described previously is also popular for users with less access to GPU resources. 

Hybrid parallel DLRM training partitions the embedding table over multiple GPUs, and places a local copy of the top DNN on each GPU. The sharded table processes an input sharded across the \textit{sample dimension}, then runs a partitioned all-gather to reaggregate the table outputs and partition them across the \textit{batch dimension} for every data parallel copy. Figure~\ref{fig:mp_dp_parallel} illustrates.

\begin{figure}[th!]
\centering
	\includegraphics[keepaspectratio=true, width=\linewidth]{images/hybrid_parallel_embedding}
	\caption{Illustration of the dataflow and execution patterns when training a DLRM using hybrid model-data parallelism across 3 GPUs with a minibatch size of 3.}
	\label{fig:mp_dp_parallel}
\end{figure}

This approach allows practitioners to benefit from both data and model parallelism within the neural architecture. The communication step is intensive, and often induces heavy overheads~\cite{mlplatformmeetup2022}, but this is generally outweighed by the benefits of parallel execution.

Overall, hybrid parallelism offers users the ability to efficiently train models by combining the benefits of different parallelization strategies when appropriate. Blended parallel techniques like sharded task parallelism and FSDP combine scalability and efficiency from the ground-up, while strategy finding and DLRM hybrid parallelism can help train model architectures that have mixed demands at different stages of the graph.




\section{Conclusion and Discussion}
\label{sec:conclusion}
In this paper, we surveyed large-model training systems research, covering various approaches ranging from heterogeneous execution to hybrid parallelism. Large-model training systems research is becoming increasingly critical as practitioners push model scales further and further. Efficiency and scalability have become key concerns for DL practitioners in a variety of domains, and the development of these systems is essential to supporting further advancements in DL.

There are several avenues yet to be explored large-model training research. Existing hybrid parallel training systems still only cover a relatively limited search space, with 3D parallel pipeline-data-tensor parallelism being the most complex configuration available to current users. Future hybrid parallel systems can be pushed further to introduce even more dimensions such as task parallelism for multi-model training and CPU spilling, potentially offering even higher scalability and efficiency.

New hardware developments will also drive new systems advances in this space. New CPU-GPU servers are being developed to support high-bandwidth NVLink communication between the CPU and GPU --- this would speed up spilling communication by nearly 60X versus PCIe 3.0x16 interconnects on Tesla GPUs~\cite{hopper2022}. TPUs and other custom hardware with increased memory capacities also offer an attractive option versus memory-limited GPUs. As the limitations and constraints of training hardware change, we will see shifts in the design and aims of large-model training software systems. 

Most current systems do not optimize explicitly for serverless execution --- all the systems outlined in this survey would work just as well in a fixed on-premise cluster as they would on cloud-provided machines. But serverless execution offers new opportunities --- and challenges. Budgeted training, for example, could allow users to specify how much they are willing to spend on a particular experiment, with automatic parallelization systems then directly determining what resources they ought to provision to support efficient training within the cost constraints. Elastic scaling for dynamic large-scale multi-model training (e.g. Hydra) could offer many exciting opportunities such as provisioning and deprovisioning resources as demand rises and wanes, or automatically adding resources to a hybrid parallel scaling dimension when it is worth it.

The large-model training systems space will continue to develop, driven forward by DL practitioners' need to explore a wider range of architectures. As compute and memory demands increase, the importance of this space will only continue to grow. The various directions of research surveyed in this paper will serve as the foundations for new upcoming developments designed to meet the changing demands of the space.
%\input{experiments}
%\input{related_work}
%\input{discussion}
%\input{acknowledgement}


% \pagebreak
\bibliography{main}
\bibliographystyle{abbrv}
%</tag>
\end{document}
